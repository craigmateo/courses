\documentclass[13.5pt]{article}
\usepackage[margin=1in]{geometry}
\usepackage{fancyhdr}
\pagestyle{fancy}
\usepackage{amssymb}
\usepackage[usenames, dvipsnames]{color}
\usepackage[T1]{fontenc}
\usepackage{biblatex}
\usepackage{amsmath}% http://ctan.org/pkg/amsmath
\newcommand{\notimplies}{%
  \mathrel{{\ooalign{\hidewidth$\not\phantom{=}$\hidewidth\cr$\implies$}}}}

\lhead{KEITH DEVLIN: Introduction to Mathematical Thinking}
\chead{}
\rhead{ASSIGNMENT 4}

\begin{document}

\begin{enumerate}
\item{Build a truth table to prove the claim I made earlier that \(\phi \Leftrightarrow \psi \) if \(\phi\) and \(\psi\) are both true or both false, and \(\phi\) and \(\psi\) is false if exactly one of \(\phi\), \(\psi\) is true and the other false.}

\begin{center}
\begin{tabular}{ c c c c c}
 \(\phi\) & \(\psi\) & \(\phi \Rightarrow \psi\) & \(\psi \Rightarrow \phi\) & \(\psi \Leftrightarrow \phi\) \\ 
\hline
 \textcolor{blue}{T} & \textcolor{blue}{T} & \textcolor{blue}{T} & \textcolor{blue}{T} & \textcolor{blue}{T}\\ 
 \textcolor{blue}{T} & \textcolor{blue}{F} & \textcolor{blue}{F} & \textcolor{blue}{T} & \textcolor{blue}{F}\\ 
 \textcolor{blue}{F} & \textcolor{blue}{T} & \textcolor{blue}{T} & \textcolor{blue}{F} & \textcolor{blue}{F}\\ 
 \textcolor{blue}{F} & \textcolor{blue}{F} & \textcolor{blue}{T} & \textcolor{blue}{T} & \textcolor{blue}{T}\\ 
\end{tabular}
\end{center}

\item{Build a truth table to show that}
\begin{center}
\((\phi \Rightarrow \psi) \Leftrightarrow (\neg \phi \vee \psi)\)
\end{center}

{is true for all truth values of \(\phi\) and \(\psi\). A statement whose truth values are all T is called a \textit{logical
validity}, or sometimes a \textit{tautology}}

\begin{center}
\begin{tabular}{ c c c c c c c}
 \(\phi\) & \(\psi\) & \(\phi \Rightarrow \psi\) & \(\neg \phi\) & \(\neg \phi \vee \psi\) & \((\phi \Rightarrow \psi) \Rightarrow (\neg \phi \vee \psi)\) & \((\neg \phi \vee \psi) \Rightarrow (\phi \Rightarrow \psi)\) \\ 
\hline
 \textcolor{blue}{T} & \textcolor{blue}{T} & \textcolor{blue}{T} & \textcolor{blue}{F} & \textcolor{blue}{T} & \textcolor{blue}{T} & \textcolor{blue}{T} \\ 
 \textcolor{blue}{T} & \textcolor{blue}{F} & \textcolor{blue}{F} & \textcolor{blue}{F} & \textcolor{blue}{F} & \textcolor{blue}{T} & \textcolor{blue}{T} \\ 
 \textcolor{blue}{F} & \textcolor{blue}{T} & \textcolor{blue}{T} & \textcolor{blue}{T} & \textcolor{blue}{T} & \textcolor{blue}{T} & \textcolor{blue}{T}\\ 
 \textcolor{blue}{F} & \textcolor{blue}{F} & \textcolor{blue}{T} & \textcolor{blue}{T} & \textcolor{blue}{T} & \textcolor{blue}{T} & \textcolor{blue}{T} 
\end{tabular}
\end{center}

\item{Build a truth table to show that}
\begin{center}
\((\phi \notimplies \psi) \Leftrightarrow (\neg \phi \vee \psi)\)
\end{center}

{is a tautology.}

\begin{center}
\begin{tabular}{ c c c c c c c c}
 \(\phi\) & \(\psi\) & \(\phi \Rightarrow \psi\) & \((\phi \notimplies \psi)\) & \(\neg \phi \vee \psi\) & \((\phi \notimplies \psi) \Rightarrow (\neg \phi \vee \psi)\) & \((\neg \phi \vee \psi) \Rightarrow (\phi \notimplies \psi)\) & \((\phi \notimplies \psi) \Leftrightarrow (\neg \phi \vee \psi)\)\\ 
\hline
 \textcolor{blue}{T} & \textcolor{blue}{T} & \textcolor{blue}{T} & \textcolor{blue}{F} & \textcolor{blue}{T} & \textcolor{blue}{F} & \textcolor{blue}{F} & \textcolor{blue}{T}\\ 
 \textcolor{blue}{T} & \textcolor{blue}{F} & \textcolor{blue}{F} & \textcolor{blue}{T} & \textcolor{blue}{F} & \textcolor{blue}{T} & \textcolor{blue}{T} & \textcolor{blue}{T}\\ 
 \textcolor{blue}{F} & \textcolor{blue}{T} & \textcolor{blue}{T} & \textcolor{blue}{F} & \textcolor{blue}{T} & \textcolor{blue}{F} & \textcolor{blue}{F} & \textcolor{blue}{T}\\ 
 \textcolor{blue}{F} & \textcolor{blue}{F} & \textcolor{blue}{T} & \textcolor{blue}{F} & \textcolor{blue}{T} & \textcolor{blue}{F} & \textcolor{blue}{F} & \textcolor{blue}{T} 
\end{tabular}
\end{center}

\item{The ancient Greeks formulated a basic rule of reasoning for proving mathematical statements. Called \textit{modus ponens}, it says that if you know  \(\phi\) and you know  \(\phi \Rightarrow \psi\), then you can conclude  \(\psi\) .}

\begin{enumerate}
\item {Construct a truth table for the logical statement}
\begin{center}
\([\phi \wedge (\phi \Rightarrow \psi)] \Rightarrow \psi\)
\end{center}

\begin{center}
\begin{tabular}{ c c c c c}
 \(\phi\) & \(\psi\) & \(\phi \Rightarrow \psi\) & \(\phi \wedge (\phi \Rightarrow \psi)\) & \([\phi \wedge (\phi \Rightarrow \psi)] \Rightarrow \psi\) \\ 
\hline
 \textcolor{blue}{T} & \textcolor{blue}{T} & \textcolor{blue}{T} & \textcolor{blue}{T} & \textcolor{blue}{T}\\ 
 \textcolor{blue}{T} & \textcolor{blue}{F} & \textcolor{blue}{F} & \textcolor{blue}{F} & \textcolor{blue}{T}\\ 
 \textcolor{blue}{F} & \textcolor{blue}{T} & \textcolor{blue}{T} & \textcolor{blue}{F} & \textcolor{blue}{T}\\ 
 \textcolor{blue}{F} & \textcolor{blue}{F} & \textcolor{blue}{T} & \textcolor{blue}{F} & \textcolor{blue}{T}\\ 
\end{tabular}
\end{center}
\item {Explain how the truth table you obtain demonstrates that \textit{modus ponens} is a valid rule of inference.}\\ \textcolor{blue}{This is a valid rule of inference because it is a tautology. In other words, \([\phi \wedge (\phi \Rightarrow \psi)] \Rightarrow \psi\) is true for every value of \(\phi\) and \(\phi \Rightarrow \psi\). Another way of looking at this is that the reasoning is valid regardless of whether the statements are true.} 

\end{enumerate}

\item{One way to prove that}
\begin{center}
\( \neg(\phi \wedge \psi) \) and \((\neg \phi) \vee (\neg \psi)\)
\end{center}

{are equivalent is to show they have the same truth table:}

\begin{center}
\begin{tabular}{ c c c c c c c}
& & & {*} & & & {*}\\
 \(\phi\) & \(\psi\) & \(\phi \wedge \psi\) & \(\neg (\phi \wedge \psi)\) & \(\neg \phi\) & \(\psi\) & \((\neg \phi) \vee (\neg \psi)\) \\ 
\hline
 \textcolor{black}{T} & \textcolor{black}{T} & \textcolor{black}{T} & \textcolor{black}{F} & \textcolor{black}{F} & \textcolor{black}{F} & \textcolor{black}{F} \\ 
 \textcolor{black}{T} & \textcolor{black}{F} & \textcolor{black}{F} & \textcolor{black}{T} & \textcolor{black}{F} & \textcolor{black}{T} & \textcolor{black}{T} \\ 
 \textcolor{black}{F} & \textcolor{black}{T} & \textcolor{black}{F} & \textcolor{black}{T} & \textcolor{black}{T} & \textcolor{black}{F} & \textcolor{black}{T}\\ 
 \textcolor{black}{F} & \textcolor{black}{F} & \textcolor{black}{F} & \textcolor{black}{T} & \textcolor{black}{T} & \textcolor{black}{T} & \textcolor{black}{T} 
\end{tabular}
\end{center}

{Since the two columns marked * are identical, we know that the two expressions are equivalent.\

Thus, negation has the affect that it changes \(\vee\) into \(\wedge\) and changes \(\wedge\) into \(\vee\). An alternative approach way to prove this is to argue directly with the meaning of the first statement:}

\begin{enumerate}
\setlength{\itemindent}{.1in}
\item[1.] \(\phi \wedge \psi\) means both \(\phi\) and \(\psi\) are true.
\item[2.] Thus \(\neg (\phi \wedge \psi)\) means it is not the case that \(\phi\) and \(\psi\) are true.
\item[3.] If they are not both true, then at least one of \(\phi\), \(\psi\) must be false.
\item[4.] This is clearly the same as saying that at least one of \(\neg \phi\) and \(\neg \psi\) is true. (By the definition of negation). 
\item[5.] By the meaning of \textit{or}, this can be expressed as \((\neg \phi)\) or \((\neg \psi)\).
\end{enumerate}

{Provide an analogous logical argument to show that \(\neg (\phi \vee \psi)\) and \((\neg \phi) \wedge (\neg \psi)\) are equivalent.


\begin{enumerate}
\setlength{\itemindent}{.1in} 
\item[1.]  \textcolor{blue} {\(\neg (\phi \vee \psi)\) means both \(\phi\) and \(\psi\) are false.}
\item[2.] \textcolor{blue}{If they are both false then neither \(\phi\) or \(\psi\) are true }
\item[3.] \textcolor{blue}{This is the same as saying that both \(\neg \phi\) and \(\neg \psi\) are true }
\item[4.] \textcolor{blue}{By the meaning of the conjunction \textit{and}, this can be expressed as \((\neg \phi)\) and \((\neg \psi)\).}
\end{enumerate}}

\item{ By a denial of a statement \(\phi\)  we mean any statement equivalent to \(\neg \phi\) . Give a useful (and hence natural sounding) denial of each of the following statements.}

\begin{enumerate}
\setlength{\itemindent}{.1in}
\item{34,159 is a prime number.}\
\textcolor{blue} {34,159 is not a prime number.}
\item{Roses are red and violets are blue.}\
\textcolor{blue} {It's not the case that roses are red and violets are blue.}
\item{If there are no hamburgers, I'll have a hot dog.}\
\textcolor{blue} {If there are no hamburgers, I may not have a hot dog.}
\item{Fred will go but he will not play.}\
\textcolor{blue} {Fred will not go or he will play.}
\item{The number \(x\) is either negative or greater than 10.}\
\textcolor{blue} {\(x\) is positive and less than or equal to 10.}
\item{We will win the first game or the second.}\
\textcolor{blue} {We will lose both games.}
\end{enumerate}

\item{Show that \(\phi \Leftrightarrow \psi\). is equivalent to \((\neg \phi) \Leftrightarrow (\neg \psi)\)  }

\begin{center}
\begin{tabular}{ c c c c c c c c c c}
 \(\phi\) & \(\psi\) & \(\phi \Rightarrow \psi\) & \(\psi \Rightarrow \phi\) & \(\psi \Leftrightarrow \phi\) & \(\neg \phi\) & \(\neg \psi\) &  \((\neg \phi) \Rightarrow (\neg \psi)\) & \((\neg \psi) \Rightarrow (\neg \phi)\) & \((\neg \phi) \Leftrightarrow (\neg \psi)\) \\ 
\hline
 \textcolor{blue}{T} & \textcolor{blue}{T} & \textcolor{blue}{T} & \textcolor{blue}{T} & \textcolor{blue}{T} & \textcolor{blue}{F} & \textcolor{blue}{F} & \textcolor{blue}{T} & \textcolor{blue}{T} & \textcolor{blue}{T} \\ 
 \textcolor{blue}{T} & \textcolor{blue}{F} & \textcolor{blue}{F} & \textcolor{blue}{T} & \textcolor{blue}{F} & \textcolor{blue}{F} & \textcolor{blue}{T} & \textcolor{blue}{T} & \textcolor{blue}{F} & \textcolor{blue}{F}\\ 
 \textcolor{blue}{F} & \textcolor{blue}{T} & \textcolor{blue}{T} & \textcolor{blue}{F} & \textcolor{blue}{F} & \textcolor{blue}{T} & \textcolor{blue}{F} & \textcolor{blue}{F} & \textcolor{blue}{T} & \textcolor{blue}{F}\\ 
 \textcolor{blue}{F} & \textcolor{blue}{F} & \textcolor{blue}{T} & \textcolor{blue}{T} & \textcolor{blue}{T} & \textcolor{blue}{T} & \textcolor{blue}{T} & \textcolor{blue}{T} & \textcolor{blue}{T} & \textcolor{blue}{T}\\ 
\end{tabular}
\end{center}

\item{ Construct truth tables to illustrate the following:}

\begin{enumerate}
\setlength{\itemindent}{.1in}
\item{\(\psi \Leftrightarrow \phi\)}
\item{\(\phi \Rightarrow (\psi \vee \theta)\)}
\end{enumerate}
\begin{center}
\begin{tabular}{c c c c c c c c}
 \(\phi\) & \(\psi\) & \(\theta\) & \(\phi \Rightarrow \psi\) & \(\psi \Rightarrow \phi\) & \(\psi \Leftrightarrow \phi\) & \(\psi \vee \theta\) & \(\phi \Rightarrow (\psi \vee \theta)\)\\ 
\hline
 \textcolor{blue}{T} & \textcolor{blue}{T} & \textcolor{blue}{T} & \textcolor{blue}{T} & \textcolor{blue}{T} & \textcolor{blue}{T} & \textcolor{blue}{T} & \textcolor{blue}{T} \\ 
 \textcolor{blue}{T} & \textcolor{blue}{F} & \textcolor{blue}{T} & \textcolor{blue}{F} & \textcolor{blue}{T} & \textcolor{blue}{F} & \textcolor{blue}{T} & \textcolor{blue}{T} \\ 
 \textcolor{blue}{F} & \textcolor{blue}{T} & \textcolor{blue}{F} & \textcolor{blue}{T} & \textcolor{blue}{F} & \textcolor{blue}{F} & \textcolor{blue}{T} & \textcolor{blue}{T} \\ 
 \textcolor{blue}{F} & \textcolor{blue}{F} & \textcolor{blue}{F} & \textcolor{blue}{T} & \textcolor{blue}{T} & \textcolor{blue}{T} & \textcolor{blue}{F} & \textcolor{blue}{T} \\ 
\end{tabular}
\end{center}

\item{Use truth tables to prove that the following are equivalent:  \(\phi \Rightarrow (\psi \wedge \theta)\) and \((\phi \Rightarrow \psi) \wedge (\phi \Rightarrow \theta)\)}\

\begin{center}
\begin{tabular}{c c c c c c c c}
 \(\phi\) & \(\psi\) & \(\theta\) & \(\psi \wedge \theta\) & \(\phi \Rightarrow (\psi \wedge \theta)\) & \(\phi \Rightarrow \psi\) & \(\phi \Rightarrow \theta\) & \((\phi \Rightarrow \psi) \wedge (\phi \Rightarrow \theta)\) \\
\hline

 \textcolor{blue}{T} & \textcolor{blue}{T} & \textcolor{blue}{T} & \textcolor{blue}{T} & \textcolor{blue}{T} & \textcolor{blue}{T} & \textcolor{blue}{T} & \textcolor{blue}{T}\\ 
 
 \textcolor{blue}{T} & \textcolor{blue}{T} & \textcolor{blue}{F} & \textcolor{blue}{F} & \textcolor{blue}{F} & \textcolor{blue}{T} & \textcolor{blue}{F} & \textcolor{blue}{F}\\ 
 
 \textcolor{blue}{T} & \textcolor{blue}{F} & \textcolor{blue}{T} & \textcolor{blue}{F} & \textcolor{blue}{F} & \textcolor{blue}{F} & \textcolor{blue}{F} & \textcolor{blue}{F}\\
 
 \textcolor{blue}{T} & \textcolor{blue}{F} & \textcolor{blue}{F} & \textcolor{blue}{F} & \textcolor{blue}{F} & \textcolor{blue}{F} & \textcolor{blue}{T} & \textcolor{blue}{F}\\ 
 
  \textcolor{blue}{F} & \textcolor{blue}{T} & \textcolor{blue}{T} & \textcolor{blue}{T} & \textcolor{blue}{T} & \textcolor{blue}{T} & \textcolor{blue}{T} & \textcolor{blue}{T}\\
  
 \textcolor{blue}{F} & \textcolor{blue}{T} & \textcolor{blue}{F} & \textcolor{blue}{F} & \textcolor{blue}{T} & \textcolor{blue}{T} & \textcolor{blue}{T} & \textcolor{blue}{T}\\ 
 
 \textcolor{blue}{F} & \textcolor{blue}{F} & \textcolor{blue}{T} & \textcolor{blue}{F} & \textcolor{blue}{T} & \textcolor{blue}{T} & \textcolor{blue}{T} & \textcolor{blue}{T}\\ 
 
 \textcolor{blue}{F} & \textcolor{blue}{F} & \textcolor{blue}{F} & \textcolor{blue}{F} & \textcolor{blue}{T} & \textcolor{blue}{T} & \textcolor{blue}{T} & \textcolor{blue}{T}\\ 
 
 
\end{tabular}
\end{center}

\item{Verify the equivalence in the previous question by means of a logical argument.}

\begin{enumerate}
\setlength{\itemindent}{.1in} 
\item[1.]  \textcolor{blue} {\(\phi \Rightarrow (\psi \wedge \theta)\) means that if \(\phi\) is true then \(\psi \wedge \theta\) is also true.}
\item[2.]  \textcolor{blue} {This means that if \(\phi\) is true then \(\psi\) is true and \(\theta\) is true.}
\item[3.]  \textcolor{blue} {It follows that \(\phi\ \Rightarrow \psi\) is the case and \(\phi\ \Rightarrow \theta\) is the case.}
\item[3.]  \textcolor{blue} {By the definition of conjunction, this is the same as \((\phi \Rightarrow \psi) \wedge (\phi\ \Rightarrow \theta)\).}
\end{enumerate}

\item{Use truth tables to prove the equivalence of \(\phi \Rightarrow \psi\) and  \((\neg\psi) \Rightarrow (\neg\phi)\). }
\begin{center}
\begin{tabular}{c c c c c c}
 \(\phi\) & \(\psi\) & \(\phi \Rightarrow \psi\) & \(\neg\phi\) &\(\neg\psi\) & \((\neg\psi) \Rightarrow (\neg\phi)\) \\
\hline
 \textcolor{blue}{T} & \textcolor{blue}{T} & \textcolor{blue}{T} & \textcolor{blue}{F} & \textcolor{blue}{F} & \textcolor{blue}{T} \\ 
 \textcolor{blue}{T} & \textcolor{blue}{F} & \textcolor{blue}{F} & \textcolor{blue}{F} & \textcolor{blue}{T} & \textcolor{blue}{F}\\ 
 \textcolor{blue}{F} & \textcolor{blue}{T} & \textcolor{blue}{T} & \textcolor{blue}{T} & \textcolor{blue}{F} & \textcolor{blue}{T}\\ 
 \textcolor{blue}{F} & \textcolor{blue}{F} & \textcolor{blue}{T} & \textcolor{blue}{T} & \textcolor{blue}{T} & \textcolor{blue}{T}\\ 
\end{tabular}
\end{center}

\item{Write down the contrapositives of the following statements:}

\begin{enumerate}
\setlength{\itemindent}{.1in}
\item{If two rectangles are congruent, they have the same area}\
\textcolor{blue} {If two rectangles do not have the same area, they are not congruent}
\item{If a triangle with sides \(a\), \(b\), \(c\) (\(c\) largest) is right-angled then \(a^2 + b^2 = c^2. \)}\
\textcolor{blue} {If a triangle has sides \(a\), \(b\), \(c\) (\(c\) largest) and it is not the case that \(a^2 + b^2 = c^2\), then the triangle is not right-angled.}
\item{If \(2^n-1\) is prime, then \(n\) is prime. }\
\textcolor{blue} {If \(n\) is not prime, then \(2^n-1\) is not prime.}
\item{If the Yuan rises, the Dollar will fall.}\
\textcolor{blue} {If the Dollar does not fall, then the Yuan will not rise.}

\end{enumerate}

\item{It is important not to confuse the contrapositive of a conditional  \(\phi \Rightarrow \psi\) with its \textit{converse} \(\psi \Rightarrow \phi\). Use truth tables to show that the contrapositive and the converse of \(\phi \Rightarrow \psi\) are not equivalent. }

\begin{center}
\begin{tabular}{c c c c c c c}
 \(\phi\) & \(\psi\) & \(\phi \Rightarrow \psi\) & \(\neg\phi\) &\(\neg\psi\) & \((\neg\psi) \Rightarrow (\neg\phi)\) & \(\psi \Rightarrow \phi\) \\
\hline
 \textcolor{blue}{T} & \textcolor{blue}{T} & \textcolor{blue}{T} & \textcolor{blue}{F} & \textcolor{blue}{F} & \textcolor{blue}{T} & \textcolor{blue}{T} \\ 
 \textcolor{blue}{T} & \textcolor{blue}{F} & \textcolor{blue}{F} & \textcolor{blue}{F} & \textcolor{blue}{T} & \textcolor{blue}{F} & \textcolor{blue}{T}\\ 
 \textcolor{blue}{F} & \textcolor{blue}{T} & \textcolor{blue}{T} & \textcolor{blue}{T} & \textcolor{blue}{F} & \textcolor{blue}{T} & \textcolor{blue}{F}\\ 
 \textcolor{blue}{F} & \textcolor{blue}{F} & \textcolor{blue}{T} & \textcolor{blue}{T} & \textcolor{blue}{T} & \textcolor{blue}{T} & \textcolor{blue}{T}\\ 
\end{tabular}
\end{center}

\item{ Write down the converses of the four statements in question 12.}

\begin{enumerate}
\setlength{\itemindent}{.1in}
\item{If two rectangles are congruent, they have the same area}\
\textcolor{blue} {If two rectangles have the same area, they are congruent.}
\item{If a triangle with sides \(a\), \(b\), \(c\) (\(c\) largest) is right-angled then \(a^2 + b^2 = c^2. \)}\
\textcolor{blue} {If a triangle has sides \(a\), \(b\), \(c\) (\(c\) largest) and\(a^2 + b^2 = c^2\), then the triangle is right-angled.}
\item{If \(2^n-1\) is prime, then \(n\) is prime. }\
\textcolor{blue} {If \(n\) is prime, then \(2^n-1\) is prime.}
\item{If the Yuan rises, the Dollar will fall.}\
\textcolor{blue} {If the Dollar falls, then the Yuan will rise.}

\end{enumerate}

\end{enumerate}
\end{document}