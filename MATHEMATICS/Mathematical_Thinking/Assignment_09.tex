\documentclass[13.5pt]{article}
\usepackage[margin=1in]{geometry}
\usepackage{fancyhdr}
\pagestyle{fancy}
\usepackage{amssymb}
\usepackage[usenames, dvipsnames]{color}
\usepackage[T1]{fontenc}
\usepackage{biblatex}
\usepackage{amssymb}
\usepackage{amsmath}% http://ctan.org/pkg/amsmath
\newcommand{\notimplies}{%
  \mathrel{{\ooalign{\hidewidth$\not\phantom{=}$\hidewidth\cr$\implies$}}}}

\lhead{KEITH DEVLIN: Introduction to Mathematical Thinking}
\chead{}
\rhead{ASSIGNMENT 9}

\begin{document}
\begin{enumerate}

\item{Express as concisely and accurately as you can the relationship between \(b|a\) and \(a/b\).}

\textcolor{blue} {\(b|a\) means \(b\) divides \(a\) or \(a=bn\) for some integer \(n\). This is equivalent to \(a/b\) or \(a\) divided by \(b\) with no remainder. So \(b|a\) iff \(a/b \in \mathbb{Z}\)}

\item{Determine whether each of the following is true or false and prove your answer.}

\begin{enumerate}
\setlength{\itemindent}{.1in}
\item{\(0|7\)}
\textcolor{blue} {False. \(0|7 \rightarrow 7/0 \in \mathbb{Z}\). But \(7/0\) is undefined.}
\item{\(9|0\)}
\textcolor{blue} {True. \(9|0 \rightarrow 0/9 \in \mathbb{Z}\). \(0/9 = 0 \in \mathbb{Z}\).}
\item{\(0|0\)}
\textcolor{blue} {False. \(0|0 \rightarrow 0/0 \in \mathbb{Z}\). But \(0/0\) is undefined.}
\item{\(1|1\)}
\textcolor{blue} {True. \(1|1 \rightarrow 1/1 \in \mathbb{Z}\). \(1/1 = 1 \in \mathbb{Z}\).}
\item{\(7|44\)}
\textcolor{blue} {False. \(7|44 \rightarrow 44/7 \in \mathbb{Z}\). But \((\not\exists n \in \mathbb{Z})(44=7n)\).}
\item{\(7|(-42)\)}
\textcolor{blue} {True. \(7|(-42) \rightarrow -42/7 \in \mathbb{Z}\). \(-42/7 = -6 \in \mathbb{Z}\).}
\item{\((-7)|(-49)\)}
\textcolor{blue} {True. \((-7)|(-42) \rightarrow -49/-7 \in \mathbb{Z}\). \(-49/-7 = 7 \in \mathbb{Z}\).}
\item{\((-7)|(-56)\)}
\textcolor{blue} {True. \((-7)|(-56) \rightarrow -56/-7 \in \mathbb{Z}\). \(-56/-7 = 8 \in \mathbb{Z}\).}
\item{\((\forall n  \in \mathbb{Z})(1|n) \)}
\textcolor{blue} {True. \((\forall n  \in \mathbb{Z})(1|n) \rightarrow (\forall n  \in \mathbb{Z})(n/1) \in \mathbb{Z}\). \(n/1 = n \in \mathbb{Z}\).}
\item{\((\forall n  \in \mathbb{N})(n|0) \)}
\textcolor{blue} {True. \((\forall n  \in \mathbb{N})(n|0) \rightarrow (\forall n  \in \mathbb{Z})(0/n) \in \mathbb{Z}\). \(0/n = 0 \in \mathbb{Z}\).}
\item{\((\forall n  \in \mathbb{Z})(n|0) \)}
\textcolor{blue} {True. \((\forall n  \in \mathbb{Z})(n|0) \rightarrow (\forall n  \in \mathbb{Z})(0/n) \in \mathbb{Z}\). \(0/n = 0 \in \mathbb{Z}\).}
\end{enumerate}

\item{Prove all the parts of the theorem in the lecture, giving the basic properties of divisibility. Namely, show that for any integers \(a\), \(b\), \(c\), \(d\) with \(a \neq 0\)}   

\begin{enumerate}
\setlength{\itemindent}{.1in}
\item{\(a|0\), \(a|a\).}\\
\textcolor{blue} {\(a|0 \rightarrow 0/a \in  \mathbb{Z}\) (True). \(a|a \rightarrow a/a=1 \in  \mathbb{Z}\) (True). }
\item{\(a|1\) if and only if \(a=\pm 1\)}\\
\textcolor{blue} {\(a|1 \rightarrow 1/a \in  \mathbb{Z} \rightarrow a=\pm 1\). \(a=\pm 1 \rightarrow 1|1\). }

\item{If \(a|b\) and \(c|d\), then \(ac|bd\) for \(c \neq 0\).}\\
\textcolor{blue} {\(a|b\) and \(c|d\) \(\rightarrow b/a \in  \mathbb{Z}\) and \(d/c \in  \mathbb{Z}\). So \(b=am\) and \(d=cn\) for integers \(m\), \(n\). \(a=\frac{b}{m}\) and \(c=\frac{d}{n}\). This means \(ac=\frac{bd}{mn}\). It follows that \(ac|bd\).}

\item{If \(a|b\) and \(b|c\), then \(a|c\) for \(a \neq 0\).}\\
\textcolor{blue} {\(a|b\) and \(b|c\) \(\rightarrow b=am\) and \(c=bn\) for integers \(m\), \(n\). So \(a=\frac{(c/n)}{m}\) and \(am=\frac{c}{n}\). \(a=\frac{b}{m}\) and \(c=\frac{d}{n}\). This means \(ac=\frac{bd}{mn}\) and \(amn=c\). \(a|c\) follows since \(mn\) is an integer.}

\item{\(a|b\) and \(b|a\) if and only if \(a=\pm b\).}\\
\textcolor{blue} {\(b=am\) and \(a=bn\) \(\rightarrow (1=mn\) and \(1=n/m) \rightarrow m=n \rightarrow [(b=a\) and \(a=b)\) or (\(b=-a\) and \(a=-b\))]. This proves the first conditional. }
\textcolor{blue} {Now assume \(a=\pm b\). \(b|a\) since \(\pm 1 \in  \mathbb{Z}\). \(a=\pm b \rightarrow b=\pm a\), so \(b=(a)(1)\) or \(b=(a)(-1)\). But this implies \(a|b\) since \(\pm 1 \in  \mathbb{Z}\), which proves the second conditional. }


\item{If \(a|b\) and \(b \neq 0\) then \(|a|\leq|b|\).}\\
\textcolor{blue} {\(a|b\), so \(b=am\) for some integer \(m\). Now assume \(|b| < |a|\). Then \(m=\frac{|b|}{|a|}\), which means \(0<m<1\). But this means \(m\) is not an integer, which is a contradiction. Therefore, \(|a|\leq|b|\).}


\item{If \(a|b\) and \(a|c\), then \(a|bx+cy\) for integers \(x\), \(y\).}\\
\textcolor{blue} {If \(a|b\) and \(a|c\), then \(b=ax\) and \(c=ay\) for integers \(x\), \(y\). So \(x=\frac{b}{a}\) and \(y=\frac{c}{a}\). \(bx=b\frac{b}{a}\) and \(cy=c\frac{c}{a}\). \(bx+cy=\frac{b^2+c^2}{a}\). But \(\b^2+c^2\) is an integer, so \(a|bx+cy\).} 



\end{enumerate}



\end{enumerate}
\end{document}
